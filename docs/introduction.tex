\chapter*{Introduction}
\addcontentsline{toc}{chapter}{Introduction}

L’optimisation des systèmes de transport et de distribution constitue aujourd’hui un enjeu majeur pour de nombreux secteurs industriels, en particulier dans le domaine de la logistique pétrolière. La distribution de carburants et de produits énergétiques implique des contraintes opérationnelles fortes, telles que la gestion de flottes de véhicules, la satisfaction de demandes multiples, la limitation des coûts et le respect de règles de sécurité strictes.

Dans ce contexte, les problèmes de tournées de véhicules (\emph{Vehicle Routing Problems}, VRP) occupent une place centrale en recherche opérationnelle. Ces problèmes visent à déterminer des itinéraires optimaux pour une flotte de véhicules afin de desservir un ensemble de clients, tout en minimisant un ou plusieurs critères de coût. Toutefois, les situations réelles présentent souvent des variantes complexes qui dépassent le cadre du VRP classique.

Le \emph{Multi-Product Vehicle Routing Problem with Changeover Cost} (MPVRP-CC) s’inscrit dans cette logique d’extension. Il modélise la distribution de plusieurs types de produits à partir de dépôts vers des stations de service, à l’aide de camions-citernes disposant de capacités limitées. Une contrainte essentielle de ce problème est qu’un véhicule ne peut transporter qu’un seul produit à la fois. Le changement de produit nécessite une opération de nettoyage de la citerne, entraînant un coût supplémentaire appelé coût de changeover. Cette contrainte introduit un compromis complexe entre distance parcourue et coût de changement de produit.

L’objectif de ce travail est d’étudier et de résoudre le MPVRP-CC en proposant une modélisation mathématique adaptée et une méthode de résolution efficace. Pour cela, une approche basée sur la programmation par contraintes est adoptée, en s’appuyant sur le solveur \emph{OR-Tools CP-SAT} développé par Google. Ce solveur permet de traiter des problèmes combinatoires complexes tout en garantissant l’optimalité ou en fournissant des solutions de haute qualité dans des temps de calcul raisonnables.

Le rapport est structuré comme suit. Le premier chapitre est consacré à la présentation détaillée du problème MPVRP-CC, de son contexte logistique et de ses principales contraintes. Le deuxième chapitre présente la modélisation mathématique du problème ainsi que les variables, contraintes et la fonction objectif retenues. Le troisième chapitre décrit la méthode de résolution mise en œuvre, incluant la construction d’une solution heuristique initiale et l’utilisation du solveur CP-SAT. Enfin, une analyse des résultats obtenus sur différentes catégories d’instances est proposée, suivie d’une conclusion générale et de perspectives.