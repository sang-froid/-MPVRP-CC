\documentclass{ifri}
\usepackage{titletoc}
\setlength{\glsdescwidth}{0.65\textwidth}

\typeMemoire{Master en Informatique}
\optionFormation{GL && SI && SIRI}
\etudiant{Prenom \textbf{Nom}}
\titreDuMemoire{Multi-Product Vehicle Routing Problem with Changeover Cost}

\dateSoutenance{-}
\anneeScolaire{2025-2026}

\encadrants{Prenom \textbf{Nom}}

\jurys{%
\begin{tabular}{llll}
	Nom et prénoms du président & Grade & Entité & Président \\
	Nom et prénoms de l'examinateur & Grade & Entité & Examinateur \\
	Nom et prénoms du rapporteur & Grade & Entité & Rapporteur \\
\end{tabular}	
}

\hypersetup{
 pdftitle={MPVRP-CC},
 pdfauthor={--},
 pdfsubject={Rapport de projet},
 pdfkeywords={VRP, MPVRP, Changeover, optimisation} 
}

\color{bookColor}

% importation du glossaire
\loadglsentries{glossaire_reduit}

\begin{document}

% =====================
% Page de garde
% =====================
\pageDeGarde
\pagecolor{white}
\selectlanguage{french}


% =====================
% Sommaire
% =====================
\pagenumbering{roman}
\setcounter{tocdepth}{0}
\startlist{toc}
\printlist{toc}{}{\chapter*{Sommaire}}
\setcounter{tocdepth}{5}


% =====================
% Résumé
% =====================
\chapter*{Résumé}
\addcontentsline{toc}{chapter}{Résumé}

Cet rapport ...
\newpage


% =====================
% Liste des figures
% =====================
\listoffigures
\newpage


% =====================
% Liste des tableaux
% =====================
\listoftables
\newpage


% =====================
% Liste des algos
% =====================
\selectlanguage{french}
\listofalgorithmes
\newpage


% =====================
% Liste des sigles / acronymes
% =====================
\setglossarystyle{altlist}
\printglossary[title=Liste des acronymes, toctitle=Liste des acronymes, type=\acronymtype]
\newpage


% =====================
% Chapitres principaux
% =====================
\pagenumbering{arabic}
\setcounter{page}{1}

\chapter*{Introduction}
\addcontentsline{toc}{chapter}{Introduction}

Le Vehicle Routing Problem (VRP) est un problème central en logistique...

\selectlanguage{french}
\fancyhead[L]{\tiny \leftmark}
\fancyhead[R]{\scriptsize \rightmark}
\fancyfoot[C]{\thepage}

\chapter{Présentation du problème}\label{chap:1}
\chapter{Présentation du problème}

\section{Description générale}
Le MPVRP-CC consiste à desservir un ensemble de clients avec un parc de véhicules, 
chaque client demandant potentiellement plusieurs produits...
 
\chapter{Modélisation du problème}\label{chap:2}
\chapter{Modélisation du problème}

\section{Données du problème}
Notations principales ...
 
\chapter{Résolution}\label{chap:3}
\chapter{Résolution}

\section{Implémentation}
Langage : Python ...

\chapter*{Conclusion et perspectives}
\addcontentsline{toc}{chapter}{Conclusion et perspectives}

Pour conclure ...
\lhead[]{} \rhead[]{} \chead[]{}


% =====================
% Bibliographie
% =====================
\addcontentsline{toc}{chapter}{Bibliographie}
\bibliographystyle{abbrv}
\bibliography{biblio}


% =====================
% Table des matières
% =====================
\newpage
\tableofcontents


\end{document}
